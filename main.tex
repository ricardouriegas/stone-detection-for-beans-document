\documentclass[conference]{IEEEtran}

% para correr / compilar 
% pdflatex main.tex
% bibtex main
% pdflatex main.tex
% pdflatex main.tex
% 

\usepackage[spanish]{babel}
\usepackage{amsmath,amssymb,amsfonts,amsthm}
\usepackage[utf8]{inputenc} % Caracteres en Español (Acentos, ñs)
\usepackage{csquotes}
\usepackage{graphicx}
\usepackage{url} % ACENTOS
\usepackage{hyperref} % Referencias
\usepackage{subfig}
\usepackage{lipsum}
\usepackage{balance} 
\usepackage{etoolbox}
\usepackage{datetime}
\usepackage{float}
\makeatletter
\patchcmd{\frontmatter@RRAP@format}{(}{}{}{}
\patchcmd{\frontmatter@RRAP@format}{)}{}{}{}
\makeatother	

\usepackage[backend=bibtex,sorting=none]{biblatex}
\setcounter{biburllcpenalty}{7000}
\setcounter{biburlucpenalty}{8000}
\addbibresource{references.bib}

% fecha
\usepackage{datetime}
\newdateformat{specialdate}{
    \twodigit{\THEDAY}-\twodigit{\THEMONTH}-\THEYEAR
}
\date{\specialdate\today}

% la sentencia \burl en las citas... 
\usepackage[hyphenbreaks]{breakurl}
\renewcommand\spanishtablename{Tabla}
\renewcommand\spanishfigurename{Figura}


\begin{document}
% Definitions
\newcommand{\breite}{0.9} %  for twocolumn
\newcommand{\RelacionFiguradoscolumnas}{0.9}
\newcommand{\RelacionFiguradoscolumnasPuntoCinco}{0.45}

%Title of paper
\title{Proyecto en Equipo U1 \\ Detección de Piedras en los Frijoles}

% Trabajo Individual
\author{
    \IEEEauthorblockN{
        Ricardo Emmanuel Uriegas Ibarra\IEEEauthorrefmark{1}
        }
    % En caso de trabajos en equipo, poner a todos los autores 
    % en estricto ORDEN ALFABETICO
    %\author{\IEEEauthorblockN{Michael Shell\IEEEauthorrefmark{1},
    %Homer Simpson\IEEEauthorrefmark{1}}
    \IEEEauthorblockA{
        \IEEEauthorrefmark{1}Ingeniería en Tecnologías de la Información\\
        Universidad Politécnica de Victoria
    }
}

\maketitle

%%%%%%%%%%%%%%%%%%%%%%%%%%%%%%%%%%%%%%%%%%%%%%%%%%%%%%%%%%%%%%%%%%%%%%%
\begin{abstract} 
    
\end{abstract}

%%%%%%%%%%%%%%%%%%%%%%%%%%%%%%%%%%%%%%%%%%%%%%%%%%%%%%%%%%%%%%%%%%%%%%%
\section{Introducción}
    % 

%%%%%%%%%%%%%%%%%%%%%%%%%%%%%%%%%%%%%%%%%%%%%%%%%%%%%%%%%%%%%%%%%%%%%%%
\section{Desarrollo Experimental}
    En este trabajo se desarrollo un sistema capaz de detectar piedras en los frijoles dada una foto (sin usar ningún tipo de red neuronal). Para ello se utiliza la librería OpenCV en Python, en combinación de la librería PyQt6 para la interfaz gráfica.
    Debido a las limitaciones de tiempo y recursos, se desarrolla el sistema solamente para 2 tipos de frijoles; pinto y negro. 

    \subsection{Características del Frijol}
    \subsubsection{Frijol Pinto}
    El frijol pinto tiene las siguientes características:
    \begin{enumerate}
        \item Frijol color cafe claro con manchas cafe oscuro.
        \item Las piedras que llegan a aparecer en este tipo de frijoles son color negro.
        \item Forma ovalada.
        \item Textura lisa.
    \end{enumerate}

    \subsubsection{Frijol Negro}
    El frijol negro tiene las siguientes características:
    \begin{enumerate}
        \item Color negro uniforme con un punto blanco que aparece en el hilum\cite{semillas}.
        \item Las piedras que llegan a aparecer en este tipo de frijoles son color gris claro (cercano al color hueso\cite{pantone}).
        \item Forma ovalada.
        \item Textura lisa.
    \end{enumerate}

    \subsection{Características de las Fotos}
    Para el desarrollo del proyecto es necesario conseguir un conjunto de pruebas que permitan verificar su correcto funcionamiento. Este conjunto de pruebas son fotos tomadas con un celular \textit{Iphone SE 2020}\cite{iphone}, esto debido a que era el celular disponible por uno de los participantes del proyecto.

    Las imagen se procesan
    


%%%%%%%%%%%%%%%%%%%%%%%%%%%%%%%%%%%%%%%%%%%%%%%%%%%%%%%%%%%%%%%%%%%%%%%
\section{Resultados}
    

%%%%%%%%%%%%%%%%%%%%%%%%%%%%%%%%%%%%%%%%%%%%%%%%%%%%%%%%%%%%%%%%%%%%%%%
\section{Conclusión}
    

%%%%%%%%%%%%%%%%%%%%%%%%%%%%%%%%%%%%%%%%%%%%%%%%%%%%%%%%%%%%%%%%%%%%%%%
\nocite{*}
\addcontentsline{toc}{section}{Referencias} 
\printbibliography
%\balance

\end{document}